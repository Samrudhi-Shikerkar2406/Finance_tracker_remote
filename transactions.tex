\documentclass[12pt,a4paper]{article}
\usepackage{graphicx}
\usepackage{amsmath}
\usepackage{geometry}
\usepackage{titlesec}
\usepackage{xcolor}
\usepackage{hyperref}
\usepackage{fancyhdr}
\geometry{margin=1in}
\hypersetup{colorlinks=true, linkcolor=blue, urlcolor=red, citecolor=blue}
\pagestyle{fancy}
\fancyhf{}
\rhead{Personal Finance Tracker}
\lhead{Samrudhi Shikerkar}
\cfoot{\thepage}

% Title formatting
\titleformat{\section}{\large\bfseries\color{blue}}{\thesection}{1em}{}
\titleformat{\subsection}{\normalsize\bfseries\color{black}}{\thesubsection}{1em}{}

% ---------------- COVER PAGE ----------------
\begin{document}
\begin{titlepage}
\centering
\vspace*{1cm}

\includegraphics[width=3cm]{university-logo-placeholder.png}\\[0.3cm]
{\small \textbf{Goa University}}\\[1.5cm]

{\Huge \bfseries Personal Finance Tracker}\\[0.4cm]
{\Large A Modular Python Project for Financial Management}\\[1cm]

\rule{\textwidth}{0.4pt}\\[0.4cm]
{\Large \textbf{Research Paper}}\\[0.2cm]
{\normalsize Under the Course: Basic Toolkit for Research}\\[0.4cm]
\rule{\textwidth}{0.4pt}\\[2cm]

\textbf{Submitted by:}\\[0.3cm]
\large \textbf{Samrudhi Shikerkar}\\
\small MSc Integrated in Data Science (2nd Year)\\
\textbf{Roll No: 21}\\[0.8cm]

\textbf{Submitted to:}\\[0.3cm]
\large \textbf{Sir Sanket Mhamal}\\
\small Faculty, Goa University\\[1.5cm]

\textbf{Date of Submission: November 2025}

\vfill
\end{titlepage}

% ---------------- PAGE 2 ----------------
% ----- ABSTRACT, INTRO, OBJECTIVES ------
\newpage
\begin{abstract}
This research paper presents the development of a collaborative software project — \textbf{Personal Finance Tracker}. The project was conducted as part of the \textbf{Basic Toolkit for Research} course, using tools such as \textbf{Git, GitHub, Markdown, and LaTeX}. The primary objective was to build a modular, Python-based application that enables users to record income and expenses, manage budgets, and visualize financial data. Each member independently developed a specific module and collaborated using Git’s distributed version control system. This paper highlights my major contribution — the \textbf{Transaction Management System}, which serves as the foundation for all financial operations.
\end{abstract}

\section{Introduction}
Financial tracking is an essential practice for both individuals and organizations. The \textbf{Personal Finance Tracker} project was designed to automate income and expense tracking while fostering teamwork and collaborative development through Git. The system comprises four main modules:
\begin{itemize}
    \item \textbf{Transactions} — Developed by Samrudhi Shikerkar (Project Owner)
    \item \textbf{Budgeting} — Developed by Falak Sardar
    \item \textbf{Visualization} — Developed by Anusha Shrivatsava
    \item \textbf{History Management} — Developed by Keziah Blossom Pereira
\end{itemize}

Each member worked on a dedicated branch, maintaining version control through GitHub. Markdown was used for documentation, while LaTeX was employed for academic reporting.

The modular architecture allowed every contributor to focus on an independent component, ensuring smooth integration in the final phase. This project, conducted under the \textbf{Basic Toolkit for Research} course, strengthened our technical, documentation, and collaborative research skills.

\section{Objectives}
The main objectives of the project were:
\begin{itemize}
    \item To design a modular Python application for income and expense management.
    \item To implement version control and collaboration using Git and GitHub.
    \item To integrate budgeting and visualization modules for financial insight.
    \item To prepare professional documentation using Markdown and LaTeX.
\end{itemize}

% ---------------- PAGE 3 ----------------
% -------- METHODOLOGY + FIGURE 1 --------
\newpage
\section{Project Workflow and Methodology}
The workflow followed these major stages:
\begin{enumerate}
    \item \textbf{Initialization:} The repository was created on GitHub and cloned by all team members.
    \item \textbf{Branch Creation:} Each member worked on a separate feature branch (\texttt{transactions}, \texttt{budget}, \texttt{history}, \texttt{visualization}).
    \item \textbf{Development:} Each module was coded and tested independently.
    \item \textbf{Push and Pull Requests:} Completed branches were pushed to GitHub, followed by pull requests for review.
    \item \textbf{Integration:} The project owner merged all branches into the master branch after successful testing. The modules were linked through the \texttt{main.py} file.
    \item \textbf{Final Version:} All members pulled the updated master branch to synchronize their local repositories.
    \item \textbf{Documentation:} Markdown and LaTeX were used for final reporting and presentation.
\end{enumerate}

\begin{figure}[h!]
\centering
\includegraphics[width=0.9\textwidth]{workflow-diagram-placeholder.png}
\caption{Collaborative Git and Branch Workflow used in the project.}
\end{figure}


% ---------------- PAGE 5 ----------------
% ------- MODULES + MY CONTRIBUTION -------
\clearpage
\section{Modules Overview}
\begin{itemize}
    \item \textbf{Transactions (Samrudhi Shikerkar):} Core module for adding, viewing, and managing financial records.
    \item \textbf{Budgeting (Falak Sardar):} Enables category-wise budget creation and tracking.
    \item \textbf{History (Keziah Blossom Pereira):} Provides date and category-based transaction filtering.
    \item \textbf{Visualization (Anusha Shrivatsava):} Displays spending patterns using pie charts.
\end{itemize}

\section{My Contribution: Transaction Management Module}
As the project owner and lead contributor, I designed and implemented the \textbf{Transaction Management System}, which forms the backbone of the entire application.

\subsection{Design and Implementation}
Using Python’s \texttt{datetime} module and object-oriented programming, transactions were represented as dictionaries containing fields such as amount, category, type, date, and notes.

\subsection{Core Formulae}
The total income ($T_{inc}$), total expense ($T_{exp}$), and balance ($B$) are calculated using:
\[
T_{inc} = \sum_{i=1}^{n} A_i, \quad T_{exp} = \sum_{j=1}^{m} A_j, \quad B = T_{inc} - T_{exp}
\]

\subsection{Functions Implemented}
\begin{itemize}
    \item \textbf{add\_transaction():} Validates and records a new transaction.
    \item \textbf{get\_all\_transactions():} Displays all stored records.
    \item \textbf{get\_transactions\_by\_category():} Filters transactions by category.
    \item \textbf{calculate\_total():} Computes total income or expense.
\end{itemize}

\subsection{Code Example}
\begin{verbatim}
tm = TransactionManager()
tm.add_transaction(1000, "Salary", "income")
tm.add_transaction(200, "Food", "expense")
print(tm.calculate_total("expense"))
\end{verbatim}

\subsection{Leadership Role}
Beyond technical development, I:
\begin{itemize}
    \item Created and maintained the GitHub repository.
    \item Managed pull requests and branch merges.
    \item Coordinated integration testing and version control.
    \item Authored the README file and this research documentation.
\end{itemize}

% ---------------- PAGE 7 ----------------
% ------ TOOLS + RESULTS + FIGURE 2 ------
\newpage
\section{Tools and Technologies Used}
\begin{itemize}
    \item Python 3.12
    \item Git and GitHub
    \item Matplotlib for visualization
    \item Markdown and LaTeX for documentation
    \item Visual Studio Code (IDE)
\end{itemize}

\section{Results and Output}
The final integrated application successfully:
\begin{itemize}
    \item Records and categorizes transactions.
    \item Calculates total income and expenses.
    \item Tracks and compares category budgets.
    \item Displays visual charts of spending.
\end{itemize}

\begin{figure}[h!]
\centering
\includegraphics[width=0.7\textwidth]{expense-piechart-placeholder.png}
\caption{Example output: Expense distribution pie chart.}
\end{figure}

% ---------------- PAGE 7 ----------------
% --------CONCLUSION + REFERENCES --------
\newpage
\section{Conclusion}
The \textbf{Personal Finance Tracker} project enhanced our understanding of Git workflows, teamwork, and software versioning. It also strengthened our Python development and research documentation skills. My primary contribution — the Transaction module — ensured accurate financial tracking and supported the integration of all other project components.

Future improvements include:
\begin{itemize}
    \item Database integration for persistent data storage.
    \item A GUI using Tkinter or Streamlit.
    \item Real-time notifications and analytical dashboards.
\end{itemize}

\section*{Acknowledgment}
I sincerely thank my teammates — Falak Sardar, Keziah Blossom Pereira, and Anusha Shrivatsava — for their cooperation and contributions, and my faculty mentor, \textbf{Sir Sanket Mhamal}, for his guidance and continuous support throughout this project.

\section*{References}
The source code for the Personal Finance Tracker project, including all modules, is publicly available on GitHub: \href{https://github.com/Samrudhi-Shikerkar2406/Finance_tracker_remote}{Personal Finance Tracker}

\end{document}